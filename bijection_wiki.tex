\documentclass[border=0.2cm]{standalone}

\usepackage{xcolor}
\usepackage{amsmath}
\usepackage{pgf, pgfplots, pgfmath, tikz}
\usepgflibrary{arrows}
\usetikzlibrary{arrows.meta}
\pgfplotsset{compat=1.18}

\begin{document}

% Macros:

\newcommand{\bijection}{
  $\text{Bijection from the natural numbers to the integers, mapping }
  2n \text{ to } -n \text{ and } 2n - 1 \text{ to } n, \text{ for } n \ge 0$
}


\newcommand{\bijectionfunction}{
$y = \frac{1}{2} \times \left[(x \mod 2) + (-1^{((x \mod 2) + 1)} \times x)\right]$
}


\newcommand{\compute}[1]{
  \pgfmathparse{(((pow(-1, (mod(#1,2) + 1)) * #1) + mod(#1,2)) * 0.5)}
}


\newcommand\maketicsandlabels{
  % Draw tick marks and X axis labels the hard way:
  \foreach \xt in {0,2,4,...,10} {
    \draw[very thin] (\xt, -0.1) -- (\xt, 0.1);
    \node[anchor=north, xshift=2mm, yshift=-1mm] at (\xt, 0) {$\xt$};
  }

  % Draw tick marks and Y axis labels the hard way:
  \foreach \yt in {-4, -2, 2, 4} {
    \draw[very thin] (-0.1, \yt) -- (0.1, \yt);
    \node[anchor=east, xshift=-2mm, yshift=0mm] at (0, \yt) {$\yt$};
  }
}


\newcommand\backgroundgrid{
  \draw[step=1cm,gray!30,ultra thin]
    (0,-6) grid (11,6);
  \draw[Stealth-Stealth] (-0.6,0) -- (11.5,0)
    node[align=right, yshift=4mm]{$X$};
  \draw[Stealth-Stealth] (0,-6.4) -- (0,6.4)
    node[align=center, xshift=-4mm]{$Y$};
}


% Main:
  \begin{tikzpicture}[
    domain=0:10,
    samples at={0,1,2,...,10}
    ]
    \tikzset{reg text/.style={black, align=left, anchor=west, xshift=0mm}
    }
    \node at (current page.center) {
      \begin{tikzpicture}
        \draw[color=blue!80] plot[] file {bijection.table};

        % Make a background grid:
        \backgroundgrid 
        % Draw the points using pgfmath computations,
        % instead of importing them from gnuplot:
        \foreach \x in {0,1,...,10} {
            \pgfmathparse{mod(\x,2)}
            \pgfmathsetmacro\resultpone{\pgfmathresult + 1}
            \pgfmathsetmacro\resultr{pow(-1,\resultpone) * \x}
            \pgfmathsetmacro\resulty{(\resultr + \pgfmathresult) * 0.5}
            \filldraw[color=black, fill=red!60] (\x, \resulty)
              circle [radius=3pt];
        }
        % Add tics and labels, er, the hard way:
        \maketicsandlabels

        % Add label indicating the math formula:
        \node[align=left] at (4.75,5.5) {\bijectionfunction};
    \end{tikzpicture}
  };  % close out the primary node
  \end{tikzpicture}

\end{document}
